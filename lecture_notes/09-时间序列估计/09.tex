%\documentclass[14pt,aspectratio=169]{beamer}
\documentclass[14pt]{beamer}
\usepackage{ctex}
\usepackage{bm}
\usepackage{color, colortbl}
\usepackage{graphicx}
\graphicspath{ {./images/} }
\setbeamertemplate{caption}{\raggedright\insertcaption\par}

\usepackage{mdframed}

\definecolor{HRed}{rgb}{1,.2,.2}
\setsansfont{Noto Sans CJK SC Light}
\setCJKsansfont[ItalicFont=Kaiti SC]{Noto Sans CJK SC Light}
\setCJKmainfont{Noto Serif CJK SC Light}
\usefonttheme[onlymath]{serif} % formulars in serif font
\usefonttheme{professionalfonts} % 防止公式间距异常,参见https://www.zhihu.com/question/55492768
\parskip=10pt

%%%%%%%%%%%%%%% Section标题页 %%%%%%%%%%%%%%%%%%%%%%%
\AtBeginSection[]{
  \begin{frame}
  \vfill
  \centering
  \begin{beamercolorbox}[sep=8pt,center,shadow=true]{title}
    \usebeamerfont{title}\insertsectionhead\par%
  \end{beamercolorbox}
  \vfill
  \end{frame}
}


%%%%%%%%%%% symbols %%%%%%%%%%%%%%
\newcommand{\mat}[1]{\bm{#1}}
\renewcommand{\vec}[1]{\bm{#1}}
\DeclareMathOperator*{\argmin}{arg\,min}

\newcommand{\MA}{\mat{A}}
\newcommand{\MI}{\mat{I}}
\newcommand{\Va}{\Vec{a}}
\newcommand{\Vy}{\vec{y}}
\newcommand{\Vx}{\vec{x}}
\newcommand{\Ve}{\vec{e}}
\newcommand{\Vw}{\vec{w}}
\newcommand{\Vt}{\vec{\theta}}
\newcommand{\SR}{\mathcal{R}}
\newcommand{\DN}{\mathcal{N}}

\let\emph\relax % there's no \RedeclareTextFontCommand
\DeclareTextFontCommand{\emph}{\color{HRed}\em}
\setbeamertemplate{headline}{}
\setbeamertemplate{navigation symbols}{}


\title{第9讲:时间序列估计}
\subtitle{动态系统建模}
\author{熊耀华}
\institute{交通工程系}

\begin{document}

\begin{frame}
    \titlepage
\end{frame}

\begin{frame}
  \frametitle{系统模型}
  \begin{description}
    \item[系统] 由互相作用的要素构成的总体。万事万物都是系统。
    \item[系统状态] 系统某一时刻的状况,可以由一组数字$\Vt$表示。
    \item[黑盒子] 系统的\emph{内部}状态、工作机制不可见。
    \item[观测值] 对\emph{外部}对系统进行测试,输入$\Vx_i$,测量对应输出$\Vy_i$。
    \item[状态估计] 根据可知的观测值$\{(\Vx_i,\Vy_i):i\in I\}$\emph{间接}估计系统状态$\Vt$
  \end{description}
  
  数据建模本质就是建立系统模型,估计系统状态。
\end{frame}

\begin{frame}
  \frametitle{静态系统和动态系统}

  \begin{description}
    \item[静态系统] 系统状态随时间\emph{不变}
    \item[动态系统] 系统状态随时间\emph{变化}
  \end{description}

  之前的数据模型,如线性回归、SVM等都静态系统模型。例如线性回归模型$y=ax+b$中,系统状态$a$、$b$
  与时间无关。
\end{frame}

\begin{frame}
  \frametitle{静态系统的贝叶斯估计}
  上一讲总的贝叶斯估计针对静态系统,写作
  \begin{equation}
    \begin{split}
      P(\Vt|\{(\Vx_i,\Vy_i):i\in I\})&\propto P(\{(\Vx_i,\Vy_i)|i\in I\}|\Vt)\cdot P(\Vt)\\
      &\propto \prod_{i\in I} P((\Vx_i,\Vy_i)|\Vt)\cdot P(\Vt)
    \end{split}
  \end{equation}
  或者说系统状态的后验概率分布等于各观测数据似然概率的\emph{连乘}。
\end{frame}

\begin{frame}
  \frametitle{动态系统的贝叶斯估计}
  \begin{itemize}
    \item 动态系统中所有观测值$\{(\Vx_i,\Vy_i):i\in I\}$称为\emph{时间序列}。
    \item 此时数据点$(\Vx_i,\Vy_i)$来自与\emph{第$i$个时段}的观察。
    \item 系统状态$\Vt$随时间变化,因此也是一个时间序列$\{\Vt_i:i\in I\}$
  \end{itemize}

  因为数据来自不同时段,简单的将数据的似然概率连乘起来没有意义。
\end{frame}

\begin{frame}
  \frametitle{估计$i$时段状态$\Vt_i$}
  将注意力集中到某一个时段$i$,该时段的状态$\Vt_i$有两个信息来源
  \begin{itemize}
    \item 第$i$时段的\emph{观测}
    \item 根据前一时段状态对本时段状态的\emph{预测}
  \end{itemize}
  将两者融合,可以用观测对预测进行\emph{修正},更可靠的估计当前状态。
\end{frame}

\begin{frame}
  \frametitle{例1:车辆定位}
  \begin{itemize}
    \item 假设一条道路离散为$10$个单元。
    \item 某车辆初始状态下位于某个单元内,每个时段向右移动一个单元。
    \item 我们通过某种设备观测车辆位置,每个时段观测一次,读数分别为${3,4,5,5,6}$。
    \item 观测设别存在误差,读数为$i$时,车辆的实际位置和概率分别为
    $$\{i-1:0.2,i:0.6,i+1:0.2\}$$
  \end{itemize}

  车辆各个时段的实际位置在哪里,概率为多少?
\end{frame}

\begin{frame}
  \frametitle{预测+修正=滤波}
  从例1总结,利用时间序列数据预测动态系统状态的流程如下:
  \begin{enumerate}
    \item 初始化,利用$(\Vx_1,\Vy_1)$估计$\Vt_1$。令$i:=2$
    \item 预测,基于$\Vt_{i-1}$预测$\hat{\Vt}_i$。(卷积)
    \item 修正,利用$(\Vx_i,\Vy_i)$修正$\hat{\Vt}_i$,得到$\Vt_i$(贝叶斯)
    \item $i:=i+1$;如果$i>N$,结束;否则回到第2步。
  \end{enumerate}

  因为这个算法最早广泛应用于信号处理领域,因此称为\emph{滤波算法}。
\end{frame}

\begin{frame}
  \frametitle{例2:机器人定位}
  \begin{itemize}
    \item 某机器人装备有激光测距设备,可测量正前方障碍物距离, 测量有显著误差。
    \item 处于一个房间中,房间几何形状已知
    \item 装备步进电机,可向任意方向前进任意距离,步进电机有显著误差
    \item 有电子罗盘,可测量绝对方位角(非必须条件)
  \end{itemize}
  设计算法,获取机器人在房间中的具体位置
\end{frame}
\end{document}