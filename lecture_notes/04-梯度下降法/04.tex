%\documentclass[14pt,aspectratio=169]{beamer}
\documentclass[14pt]{beamer}
\usepackage{ctex}
\usepackage{bm}
\usepackage{color, colortbl}
\definecolor{HRed}{rgb}{1,.2,.2}
%\usepackage{xeCJK} % important! Without this Chinese fonts won't show
%\usepackage{xeCJKfntef}
\setsansfont{Noto Sans CJK SC Light}
\setCJKsansfont[ItalicFont=Kaiti SC]{Noto Sans CJK SC Light}
\setCJKmainfont{Noto Serif CJK SC Light}

\usefonttheme[onlymath]{serif} % formulars in serif font
\usefonttheme{professionalfonts} % 防止公式间距异常,参见https://www.zhihu.com/question/55492768
\parskip=10pt

%\newcommand{\mat}[1]{\bm{#1}}
\newcommand{\mat}[1]{\bm{#1}}
\renewcommand{\vec}[1]{\bm{#1}}
\DeclareMathOperator*{\argmin}{arg\,min}

\newcommand{\MA}{\mat{A}}
\newcommand{\MI}{\mat{I}}
\newcommand{\Va}{\Vec{a}}
\newcommand{\Vy}{\vec{y}}
\newcommand{\Vx}{\vec{x}}
\newcommand{\Ve}{\vec{e}}
\newcommand{\Vt}{\vec{\theta}}
\newcommand{\SR}{\mathcal{R}}

\let\emph\relax % there's no \RedeclareTextFontCommand
\DeclareTextFontCommand{\emph}{\color{red}\em}
\setbeamertemplate{headline}{}
\setbeamertemplate{navigation symbols}{}

%%%%%%%%%%%%%%% Section标题页 %%%%%%%%%%%%%%%%%%%%%%%
\AtBeginSection[]{
  \begin{frame}
  \vfill
  \centering
  \begin{beamercolorbox}[sep=8pt,center,shadow=true]{title}
    \usebeamerfont{title}\insertsectionhead\par%
  \end{beamercolorbox}
  \vfill
  \end{frame}
}
%%%%%%%%%%%%%%% 正文开始 %%%%%%%%%%%%%%%%%%%%%%%

\title{第4讲:梯度下降法}
\subtitle{线性回归的数值解}
\author{熊耀华}
\institute{交通工程系}

\begin{document}

\begin{frame}
    \titlepage
\end{frame}

\section{线性回归数值解}

\begin{frame}
  \frametitle{解析解}
  之前的例子通过公式推导求解
  $$\Vt^*=\argmin L(\Vt)$$
  得到的公式称为\emph{解析解},例如
  \begin{equation*}
    \MA^T\MA\Vt^*=\MA^T\Vy \qquad (\MA^T\MA+\lambda\MI)\Vt^*=\MA^T\Vy
  \end{equation*}
\end{frame}

\begin{frame}
  \frametitle{数值解}
  解析解虽然在数学上更优雅,但实际应用中往往通过\emph{数值方法}求\emph{数值解}
  \begin{itemize}
    \item 很多模型的损失函数没有解析解,只能求数值解
    \item 对于大量数据,数值解可以用\emph{精度换时间},快速得到低精度结果
  \end{itemize}
\end{frame}

\begin{frame}
  \frametitle{梯度下降法}
  \emph{梯度下降法}(Gradient Descent)是最小化问题中的常用数值方法,基本思路如下
  \begin{enumerate}
    \item 将目标函数$L(\Vt)$看作起伏不平的地面,要找到地面最低点
    \item 从随机位置开始
    \item 观察该位置上不同方向的坡度,选择坡度\emph{下降最快}的方向,前进一定距离
    \item 新位置坡度是否为$0$?
    \item 是则到达最低点,结束;否则回到第3步
  \end{enumerate}
\end{frame}
  
\begin{frame}
  \frametitle{梯度方向}
  连续函数$L(\Vt)$的\emph{梯度}$\nabla L$(读做nabla L)是所有偏导数构成的向量
  \begin{equation}
    \nabla_{\Vt} L(\Vt) = \begin{bmatrix}
      \frac{\partial L}{\partial\theta_1}(\Vt)\\
      \frac{\partial L}{\partial\theta_2}(\Vt)\\
      \frac{\partial L}{\partial\theta_3}(\Vt)\\
      \vdots
    \end{bmatrix}
  \end{equation}
  用梯度可以估计函数自变量变化$\Delta\Vt$时,函数值的变化
  $$\Delta L ={{\nabla L(\Vt)}^T\cdot\Delta\Vt}$$
\end{frame}

\begin{frame}
  \frametitle{梯度反方向下降最快}
  可以证明,如果$\Delta\Vt$长度一定,方向可变,那么当$\Delta\Vt$与梯度方向一致时$\Delta L$最大。
  
  提示:向量点积有如下性质
  $$\Vx\cdot\Vy=\|\Vx\|\|\Vy\|\cos\alpha$$
  其中$\alpha$是两个向量的夹角。

  因此梯度方向是函数值上升最快的方向,而梯度的反方向是函数值\emph{下降最快}的方向。
\end{frame}

\begin{frame}
  \frametitle{梯度下降迭代公式}
  基于以上分析,梯度下降法可以表示为迭代公式
  \begin{equation}
    \Vt^{n+1}=\Vt^n-\eta\cdot\nabla_{\Vt} L(\Vt^n)
  \end{equation}
  其中$\eta$(eta)称为\emph{学习速率}(Learning Rate),控制参数$\Vt$的迭代更新速度
\end{frame}

\begin{frame}
  \frametitle{线性回归的梯度}
      线性回归的损失函数是
      \begin{equation*}
        L(\Vt)=\frac{1}{m}(\MA\Vt-\Vy)^T(\MA\Vt-Vy)
      \end{equation*}
      对$\Vt$求偏导有
      \begin{equation}\label{eq:gradient_ls}
        \begin{split}
        \nabla_{\Vt}L(\Vt)&=\frac{1}{m}[\MA^T\MA\Vt-\MA^T\Vy]\\
        &=\frac{1}{m}\MA^T(\MA\Vt-\Vy)\\
        &=\boxed{\frac{1}{m}\MA^T(\hat{\Vy}-\Vy)}
        \end{split}
      \end{equation}
\end{frame}

\begin{frame}
  \frametitle{Ridge回归的梯度}
      Ridge回归的损失函数是
      \begin{equation*}
        L(\Vt)=\frac{1}{m}(\MA\Vt-\Vy)^T(\MA\Vt-Vy)+\lambda\Vt^T\Vt
      \end{equation*}
      对$\Vt$求偏导有
      \begin{equation}
        \begin{split}
        \nabla_{\Vt}L(\Vt)&=(\frac{1}{m}\MA^T\MA+\lambda\MI)\Vt-\frac{1}{m}\MA^T\Vy\\
        &=\boxed{\frac{1}{m}\MA^T(\hat{\Vy}-\Vy)+\color{red}{\lambda\Vt^T\Vt}}
        \end{split}
      \end{equation}
      
\end{frame}
\begin{frame}
  \frametitle{梯度下降法常见问题}
  使用梯度下降法容易遇到的问题有
  \begin{itemize}
    \item 学习速率过大,\emph{震荡}而不收敛
    \item 学习速率过小,收敛慢
    \item 遇到\emph{局部最优值},落入“陷阱”
    \item 遇到\emph{平台},收敛慢
  \end{itemize}

\end{frame}

\begin{frame}
  \frametitle{练习}
  使用梯度下降法求解线性回归,画出损失函数和迭代次数的变化曲线,也称为\emph{学习曲线}
\end{frame}

\begin{frame}
  \frametitle{优化梯度计算速度}
  之前的梯度计算因为要用到所有的数据$\Vx$、$\Vy$,因此称为\emph{批处理}(batch)。
  
  当数据量很大时,批处理计算速度较慢,需要优化。两个常见的思路是
  \begin{itemize}
    \item 随机(stochastic)梯度下降,每次迭代随机选取\emph{一个}数据计算下降方向
    \item 分批(mini-batch)梯度下降,每次迭代随机选取\emph{一组}数据计算下降方向
  \end{itemize}
\end{frame}

\begin{frame}
  \frametitle{随机梯度下降}
  以线性回归为例,某次迭代中随机选择了第$n$个数据$(x_n, y_n)$。此时损失函数为
  \begin{equation}
    L(\Vt)={(y_n-\Va^*_n\cdot\Vt)}^2
  \end{equation}
  其中$\Va^*_n$是$\MA$的第$n$对$\Vt$求偏导有
  \begin{equation}
    \begin{split}
      \nabla_{\Vt}L(\Vt) &=\Va^*_n{\Va^*_n}^T\Vt+\Va^*_ny_n\\
      &=(x_n^2+1)\Vt+(x_n+1)y_n
    \end{split}
  \end{equation}
\end{frame}

\begin{frame}
  \frametitle{分批梯度下降}
  分批(mini-batch)梯度下降是batch和stochastic之间的妥协。
  
  某次迭代中随机选择一组$n$个数据,损失函数为
  \begin{equation}
    L(\Vt)={(y_n-\MA^{(n)}\cdot\Vt)}^2
  \end{equation}
  其中$\MA^{(n)}$是$\MA$的随机选择$n$行构成的\emph{子矩阵}(sub matrix),偏导为
  \begin{equation}
      \nabla_{\Vt}L(\Vt)= \MA^{(n)} {\MA^{(n)}}^T\Vt+\MA^{(n)}\Vy^{(n)}
  \end{equation}
\end{frame}

\begin{frame}
  \frametitle{练习}
  用stochastic和mini-batch方法求解线性回归,画出$\Vt$的\emph{相空间}图。
\end{frame}

\begin{frame}
  \frametitle{Lasso回归}
 Least Absolute Shrinkage and Selection Operator Regression,简称\emph{Lasso}回归。
 
 Ridge回归的损失函数是
 $$L(\Vt)=\Ve^T\Ve+\lambda\Vt^t\Vt=\Ve^T\Ve+\lambda{\color{red}\sum\theta_i^2}$$
而Lasso回归的损失函数是
\begin{equation}
L(\Vt)=\Ve^T\Ve+\lambda{\color{red}\sum|\theta_i|}
\end{equation}
\end{frame}

\begin{frame}
  \frametitle{Ridge、Lasso和范数}
  Ridge回归的\emph{正则项}是参数矢量$\Vt$各元素的\emph{平方和},Lasso回归中的正则项是元素的\emph{绝对值之和}。

  本质上是两种计算矢量“长度”的方法,是矢量\emph{范数}(norm)的特殊形式。对于矢量$\Vx={x_1,\ldots,x_n}$范数的一般定义为
  \begin{equation}
    \|\Vx\|_p=\left(\sum_{i=1}^n{|x_i|}^p\right)^{1/p}
  \end{equation}
\end{frame}

\begin{frame}
  \frametitle{常用范数}
  范数$\|\|_p$的参数$p$可以取任何正数。但常见的取值和他们的性质如下
  \begin{description}
    \item[$\|\Vx\|^0$] 矢量中非0元素的个数,严格来说不是范数
    \item[$\|\Vx\|^1$] 绝对值之和,带来稀疏性
    \item[$\|\Vx\|^2$] 平方和,又称\emph{欧几里德长度},符合我们的几何直觉
    \item[$\|\Vx\|^\infty$] 最大元素$\max\{x_1,\ldots,x_n\}$
  \end{description}
\end{frame}

\begin{frame}
  \frametitle{Lasso带来参数稀疏性}
  Lasso回归使用$1$次范数作为正则项,目的是使参数矢量$\Vt$\emph{稀疏}。

  $1$次正则项的几何形状可以解释稀疏性
  
  稀疏的好处是,大量参数为$0$,对应的属性对模型预测不产生影响。因此通过Lasso回归可以识别\emph{相关属性}和\emph{无关属性}。
\end{frame}

\begin{frame}
  \frametitle{Ridge和Lasso的混合}
  Ridge回归具有\emph{稳定性}而Lasso回归具有\emph{稀疏性}。Elastic net是两者混合,损失函数定义为
  \begin{equation}
    L(\Vt)=\Ve^T\Ve+\eta(r\sum|\theta_i|+(1-r)\frac{1}{2}\sum\theta_i^2)
  \end{equation}
  其中$r$表示Ridge和Lasso的相对权重。

  适当的混合比例可以使模型同时具有两者的良好性质。
\end{frame}

\section{分类模型}

\begin{frame}
  \frametitle{分类模型}
  前面的回归模型中,我们建立了函数关系$f:\Vx\to\Vy$用于从$\Vx$预测$\Vy$,此时$\Vy\in \SR$可以取任何值。

  现在我们希望建立函数,判断$\Vx$\emph{是否}满足某种条件。此时$\Vy$应当表示满足条件的概率,$\Vy\in[0,1]$

  为此我们可以构造一个函数$g:\SR\to[0,1]$,与$f$组成复合函数
  \begin{equation}
    g\circ f:\Vx\to[0,1]
  \end{equation}
\end{frame}

\begin{frame}
  \frametitle{Logistic函数}
  满足$g:\SR\to[0,1]$的函数中,最常用的称为Logisitic函数(或者Sigmoid函数)
  \begin{equation}
    \sigma(t)=\frac{1}{1+e^{-t}}
  \end{equation}
  从函数图像可以看出
  \begin{itemize}
    \item $\lim_{t\to\infty}\sigma(t)=1$
    \item $\lim_{t\to-\infty}\sigma(t)=0$
    \item 关于$(0, 0.5)$对称
  \end{itemize}
  

\end{frame}
\begin{frame}
  \frametitle{Logistic回归}
  用以下复合函数拟合$\Vx$,$\Vy$
  \begin{equation}
    \sigma(\MA\Vt)
  \end{equation}
  其中,$\Vy$矢量中的元素只能取$\{0,1\}$,分别表示两种\emph{可能的类别};$\MA=\begin{bmatrix}
    \Vx & \vec{1}
  \end{bmatrix}$

  通过选取最佳的参数$\Vt^*$使\emph{预测值}$\hat{\Vy}=\sigma(\MA\Vt^*)$与\emph{观测值}$\Vy$\emph{最接近}。如何定义“接近”?
\end{frame}

\begin{frame}
  \frametitle{Logistic函数的导数}
  梯度下降法求解时需要Logistic函数的导数。Logistic函数的导数具有\emph{特别形式}
  \begin{equation}
    \sigma'(t)=\sigma(t)(1-\sigma(t))
  \end{equation}
  请验证?
  
  提示:用求导规则
  $$\left(\frac{1}{u}\right)'=\frac{u'}{u^2}$$
\end{frame}
\begin{frame}
  \frametitle{如何比较概率?}
  线性回归中将接近定义为\emph{残差的平方和},这个定义不适合Logistic回归。因为Logistic回归中的$\Vy$表示\emph{概率},而概率的接近程度并不是线性的。
  
  考虑以下三种情况,交通事故发生的概率为$\{0.01, 0.001, 0.0001\}$。概率数字变化不大,但表示的情况差别很明显,三种情况的安全性各差了一个\emph{数量级}。

  要体现数量级的差距,用概率的\emph{对数函数}更合适。例如$\{\log0.01, \log0.001, \log0.0001\}$三个值之间距离相等。
\end{frame}

\begin{frame}
  \frametitle{Logistic回归的损失函数}
  基于对概率比较的分析,Logistic回归的函数定义如下。对\emph{某个}数据点$(x_i, y_i)$
  \begin{equation}\label{eq:logistic_loss_seperated}
    l_i(\Vt)=\left\{\begin{array}{ll}
      -\log(\hat{y}_i) & \text{如果}y_i=1\\
      -\log(1-\hat{y}_i) & \text{如果}y_i=0
    \end{array}\right.
  \end{equation}
  这个定义有以下特点
  \begin{itemize}
    \item 当$y_i=1$时数据点属于$1$类,此时如果预测值$\hat{y}_i=1$则完美符合,没有损失
    \item 当$y_i=0$时数据点属于$0$类,此时如果预测值$\hat{y}_i=0$则完美符合,没有损失
    \item 损失$l\ge0$
  \end{itemize}
\end{frame}

\begin{frame}
  \frametitle{Logistic回归的损失函数}
  公式\ref{eq:logistic_loss_seperated}是分段函数,使用不便。我们利用一个\emph{小技巧}把它们组合起来
  \begin{equation}\label{eq:logistic_loss_combined}
    l_i(\Vt)=-[y_i\cdot\log(\hat{y}_i)+(1-y_i)\cdot\log(1-\hat{y_i})]
  \end{equation}

  可以证明公式\ref{eq:logistic_loss_combined}和\ref{eq:logistic_loss_seperated}等价(试一下)

\end{frame}

\begin{frame}
  \frametitle{Logistic回归的损失函数}
  考虑所有$m$个数据点,总损失函数为
  \begin{equation*}
    \begin{split}
    L(\Vt)&=\frac{1}{m}\sum_{i=1}^m l_i(\Vt)\\
    &=-\frac{1}{m}\sum_{i=1}^m[y_i\cdot\log(\hat{y}_i)+(1-y_i)\cdot\log(1-\hat{y_i})]
    \end{split}
  \end{equation*}
  进一步整理成矩阵形式有
  \begin{equation}\label{eq:logistic_loss}
    \boxed{-\frac{1}{m}[\Vy^T\cdot\log(\sigma(\MA\Vt))+(1-\Vy)^T\cdot\log(\vec{1}-\sigma(\MA\Vt))]}
  \end{equation}
\end{frame}

\begin{frame}
  \frametitle{Logsitic回归的梯度}
  对公式\ref{eq:logistic_loss}求偏导,得到Logistic回归的梯度为
  \begin{equation}
    \begin{split}
    \nabla_{\Vt}L(\Vt)&=\frac{1}{m}\MA^T(\sigma(\MA\Vt)-\Vy)\\
    &=\frac{1}{m}\MA^T(\hat{\Vy}-\Vy)
    \end{split}
  \end{equation}
  与线性回归的梯度公式\ref{eq:gradient_ls}比较,两者形式\emph{高度相似}

\end{frame}

\begin{frame}
  \frametitle{练习}
  梯度下降法进行二个属性数据的分类,画出属性空间图。注意边界
  
\end{frame}

\begin{frame}
  \frametitle{从两类到多类}
  Logistic回归只能解决分两类的问题,扩展到多类上称为Softmax模型。

  假设存在$k$个类别,每个类别都有一组\emph{打分系数}$\Vt_j,1\leq j\leq k$。
  某组样本$(\Vx_i, y_i)$与类别吻合的分数分别为
  \begin{equation}
    s_{ij}=\Vt_j^T\cdot\Vx_i
  \end{equation}
\end{frame}

\begin{frame}
  \frametitle{分数到概率}
  根据分数,最终样本$i$属于类别$j$的概率为
  \begin{equation}
    \hat{p}_{ij}=\frac{e^{s_{ij}}}{\sum_{j=1}^k e^{s_{ij}}}
  \end{equation}

  这个函数称为softmax,相对于“hard” max。
  \begin{itemize}
    \item “hard” max从一组数中挑出最大的数字,让他的对应权重为$1$,其余为$0$
    \item Softmax扩大最大数字的权重,如果\emph{反复应用},最大元素的权重趋于$1$
  \end{itemize}
\end{frame}

\begin{frame}
  \frametitle{Logistic回归是Softmax的特例}
  当类别数量$k=2$时,Softmax和Logistic回归等价。

  TODO证明
\end{frame}

\begin{frame}
  \frametitle{Softmax的损失函数}
  使用类似于Logistic回归的分析方法,可以构造出Softmax的损失函数
  \begin{equation}
    L(\Vt)=-\frac{1}{m}\sum_{i=1}^m\sum_{j=i}^k y_{ij}\log(\hat{p}_{ij})
  \end{equation}
  其中$y_{ij}$表示第$i$个样本是否属于第$j$类,在$\{0,1\}$内取值。
  这个损失函数又称为\emph{交叉熵}函数。
\end{frame}

\begin{frame}
  \frametitle{交叉熵}
  \emph{熵}(entropy)和\emph{交叉熵}(cross entropy)都是来自于信息论的概念。其中熵的定义
  是,对于分布函数$p(x)$,熵表示包含信息的多少,定义为
  \begin{equation}
    E(p)=-\sum_x p(x)\log({\color{red}p(x)})
  \end{equation}
  相应的交叉熵用于比较两个不同分布$p(x)$,$q(x)$差别大小,定义为
  \begin{equation}
    H(p,q)=-\sum_x p(x)\log({\color{red}q(x)})
  \end{equation}
\end{frame}

\begin{frame}
  \frametitle{交叉熵}
  $\{y_{ij}:1\leq j\leq k\}$和$\{\hat{y}_{ij}:1\leq j\leq k\}$是两个不同的分布,
  因此Softmax的损失函数符合交叉熵的定义。

  

\end{frame}
\end{document}