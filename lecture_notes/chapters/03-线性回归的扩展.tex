\chapter{扩展线性回归}

上一章我们从路段交通流参数建模问题入手,接触到了线性回归模型,用直线拟合密度--速度之间的函数关系。
直线拟合应用范围很广泛,但是现实生活中很多变量之间的函数关系很可能不是直线,线性回归是否还能适用?
如果能够适用,那么“\emph{线性}”的含义并不等价于“直线”,具体是什么?本章我们回答这些问题。

从本章开始为了讨论的\emph{一般性}我们引入一套新的字母命名规则。自变量用$x$表示、因变量用$y$、自变量的权重系数用$w$,其他系数用$b$。
根据新规则上一章讨论的线性回归模型可以从\cref{eq:linear-kv}转写成以下形式
\begin{equation*}
    y=w\cdot x+b
\end{equation*}

\section{高次多项式线性回归}
我们现在来拟合流量--密度关系。如\cref{fig:empirical-qkv}右侧所示,随着密度上升路段流量呈现先上升后下降的关系,这种情况用直线拟合显然不合适。能够满足先升后降的函数中最简单的是二次多项式函数,也就是抛物线
\begin{equation}
    y=w_2 x^2 + w_1 x + b
\end{equation}
其中$w_2$、$w_1$、$b$都是未知参数。

\subsection{抛物线}

我们是否可以用上一章的思路来确定参数呢?答案是可以的。假设我们有$n$个数据点$\{x_i,y_i\},i=1\ldots n$,如果抛物线通过所有数据点,以下方程组应该成立
\begin{equation}
    \begin{bmatrix}
        y_1\\
        y_2\\
        \vdots\\
        y_n
    \end{bmatrix}=
    \begin{bmatrix}
        x_1^2 & x_1 & 1\\
        x_2^2 & x_2 & 1\\
        \vdots & \vdots & \vdots\\
        x_n^2 & x_n & 1\\
    \end{bmatrix}
    \begin{bmatrix}
        w_2\\
        w_1\\
        b
    \end{bmatrix}
\end{equation}
用向量和矩阵符号缩写后得到
\begin{align}
    \Vy&=\begin{bmatrix}
        \Vx^2 &\Vx &\Vi
    \end{bmatrix}
    \begin{bmatrix}
        w_2\\
        w_1\\
        b
    \end{bmatrix}\\
    &=\MA\Vtheta
\end{align}
其中$\MA$表示矩阵,$\Vtheta$表示所有未知参数。

沿用之前的分析,可以推测我们有$n=3$个数据点时,$\MA$是$3×3$正方形矩阵,方程一般有解;当$n≧4$时方程一般无解。此时我们依然可以利用最小二乘法,在方程左右两边同时乘以$\MA^T$然后求解近似解
\begin{equation}
    \MA^T\MA\Vtheta^*=\MA^T\Vy
\end{equation}

\begin{example}
    基于以下数据,用抛物线拟合\emph{密度}和流量的关系。
    注意数据中密度没有直接给出,需要用公式$q=k·v$计算得到。
    \begin{center}
        \begin{tabular}{ccc}
            \toprule
            数据编号 & 速度(公里/小时) & 密度(辆/公里) \\
            \midrule
            1 & 80 & 10 \\
            2 & 50 & 100 \\
            3 & 20 & 150\\
            4 & 40 & 80\\
            \bottomrule
        \end{tabular}
    \end{center}
\end{example}

从这个例子可以看出“线性”一词的含义,不是拟合函数关于自变量$x$线性,而是关于参数$\Vtheta$线性。
这个定义极大的扩展了线性回归的应用范围,任何可以分离参数$\Vtheta$的函数都可以作为拟合函数。
\begin{example}
    函数$y=a^x+b$是否符合线性回归要求?
\end{example}
\begin{solution}
    表面上看由于$a^x$的存在函数不符合线性回归的要求,但是可以通过两边取对数分离参数$a$和$x$
    \begin{align*}
        \log{y}&=x\log{a}+\log{b}\\
    \end{align*}
    此时可以认为$x$是自变量,$\log{y}$是因变量,$\log{a}$、$\log{b}$是参数,符合线性回归要求。
\end{solution}
\subsection{过拟合和正则化}

从直线和抛物线函数拟合的分析中我们可以总结出一般规律。对于$n$个数据点,要\emph{完美拟合}我们需要用$n-1$次多项式,此时参数可以直接求解;当多项式次数少于$n-1$时我们只能\emph{近似拟合},参数用最小二乘法近似求解。需要注意的是在实际数据处理中我们一般不会用高次多项式做完美拟合,而是有意识的选取低次多项式近似拟合。

这样做的原因有两个。一是高次多相式\emph{数值稳定性}很差。所谓数值稳定性是指当计算过程中某个步骤出现的微小数值误差在最终结果中会放大还是缩小。
如果放大,则我们说数值稳定性差;反之则稳定性好。
所有计算机在做数值计算时由于设计局限性都会产生微小的误差%
\sidenote{详情见计算机浮点运算规范,例如IEEE 754。}
,如果程序数值稳定性好,最终结果虽然也有误差,但可以接受。而数值稳定性差的程序会把误差急剧放大,最后的结果毫无意义。

另一个原因是低次多项式可以过滤数据中的噪声。我们的数据在采集过程中都会混入\emph{噪声},因此完美的拟合数据反而不能完美的符合实际情况,这种情况我们称为\emph{过拟合(overfit)}。
合理选择多项式次数,拟合时与数据的误差反而能抵消数据和实际情况的误差,让结果更符合实际情况。如果因为某种原因必须使用高次多项式,一般需要配合参数的\emph{正则化(Regularization)},即缩小参数的取值范围,详情见后续章节。

\section{高维度线性回归}

线性回归方法也可以向高维度数据扩展。前面的分析中每个数据点的自变量$x$和因变量$y$都是\emph{标量},也就是说自变量和因变量之间是“一对一”的关系。但现实世界中自变量和因变量都可以是\emph{矢量},两者之间的关系更加复杂。例如,决定交通流速度的不只是车辆密度,实际上可能还有道路能见度、风速等等各种指标;这是“一对多”关系。卫星定位问题中车辆位置分为横坐标和纵坐标两个分量,共同由五颗卫星的距离决定,这是“多对多”关系。

高维度线性回归是指自变量或因变量是矢量的线性回归,又称为\emph{多元线性回归}。
这里我们主要介绍自变量为标量,因变量为矢量的“一对多”形式。
假设我们有$n$个数据点$\{\Vx_i,y_i\}, i=1\cdot n$,其中自变量$\Vx_i$是$m$维矢量
\begin{equation*}
    \Vx_i=
    \begin{bmatrix}
        x_{i1}\\
        x_{i2}\\
        \vdots\\
        x_{im}\\
    \end{bmatrix}
\end{equation*}

假设自变量与因变量之间是线性关系,第$i$组数据的关系写作
\begin{align}
    y_i &= w_1x_{i1}+w_2x_{i2}+\cdots+w_mx_{im}+b\nonumber\\
    &=\begin{bmatrix}
        x_{i1} & x_{i2} & \cdots & x_{im}
    \end{bmatrix}
    \begin{bmatrix}
        w_1\\
        w_2\\
        \vdots\\
        w_m
    \end{bmatrix}+b\nonumber\\
    &=\Vx_i^T\Vw+b
\end{align}
其中$\Vw$是权重系数构成的向量。

所有$n$组数据的关系共同可以写作
\begin{align}
    \begin{bmatrix}
        y_1\\
        y_2\\
        \vdots\\
        y_n
    \end{bmatrix}
    &=
    \begin{bmatrix}
        \Vx_1^T\\
        \Vx_2^T\\
        \vdots\\
        \Vx_n^T
    \end{bmatrix}\Vw+
    \begin{bmatrix}
        b\\
        b\\
        \vdots\\
        b
    \end{bmatrix}\nonumber\\
    \intertext{整理成矩阵形式}
    \Vy&=\MX\Vw+\Vi b\\
    &=\begin{bmatrix}
        \MX & \Vi
    \end{bmatrix}
    \begin{bmatrix}
        \Vw\\
        b
    \end{bmatrix}\nonumber\\
    &=\MA\Vtheta
\end{align}
其中$\MA$是由$\MX$和$\Vi$拼接的矩阵,$\Vy$是所有因变量$y_i$构成的向量,$\MX$是由所有自变量$x_{ij}$数据构成的矩阵。
\begin{equation*}
    \Vy=\begin{bmatrix}
        y_1\\
        y_2\\
        \vdots\\
        y_n
    \end{bmatrix}
    \text{,}\quad
    \MX=\begin{bmatrix}
        x_{11} & x_{12} & \cdots & x_{1m}\\
        x_{21} & x_{22} & \cdots & x_{1m}\\
        \vdots & \vdots & \vdots & \vdots\\
        x_{n1} & x_{n2} & \cdots & x_{nm}\\
    \end{bmatrix}
\end{equation*}

\begin{example}
    多元线性回归。
\end{example}

\section{数据模型的一般形式}

从多元线性回归我们可以进一步推广,得到数据模型的一般形式。
\begin{equation}
    \Vy=f(\Vx;\Vtheta)
\end{equation}
或者
\begin{equation}
    f_{\Vtheta}:\Vx\to\Vy
\end{equation}
这两个公式表达相同的意思,数据模型是一个函数$f$,可以根据输入数据,也就是自变量$\Vx$预测输出数据,也就是因变量$\Vy$。数据模型函数有一些可以调整的参数,共同记作$\Vtheta$。

线性回归是这个数据模型的一种特殊形式,要求$\Vx$和$\Vy$都是\emph{连续变量},且$f$是相对$\Vtheta$的线性函数。
进一步细分,如果$\Vx$的维度大于等于二维,我们得到多元线性回归;如果只有一维,那我们得到一元线性回归。
如果$\Vy$不是连续变量而是\emph{离散变量},那我们得到分类模型,见后续章节。