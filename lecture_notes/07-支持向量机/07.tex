%\documentclass[14pt,aspectratio=169]{beamer}
\documentclass[14pt]{beamer}
\usepackage{ctex}
\usepackage{bm}
\usepackage{color, colortbl}
\definecolor{HRed}{rgb}{1,.2,.2}
%\usepackage{xeCJK} % important! Without this Chinese fonts won't show
%\usepackage{xeCJKfntef}
\setsansfont{Noto Sans CJK SC Light}
\setCJKsansfont[ItalicFont=Kaiti SC]{Noto Sans CJK SC Light}
\setCJKmainfont{Noto Serif CJK SC Light}

\usefonttheme[onlymath]{serif} % formulars in serif font
\usefonttheme{professionalfonts} % 防止公式间距异常,参见https://www.zhihu.com/question/55492768
\parskip=10pt

%\newcommand{\mat}[1]{\bm{#1}}
\newcommand{\mat}[1]{\bm{#1}}
\renewcommand{\vec}[1]{\bm{#1}}
\DeclareMathOperator*{\argmin}{arg\,min}

\newcommand{\MA}{\mat{A}}
\newcommand{\MI}{\mat{I}}
\newcommand{\Va}{\Vec{a}}
\newcommand{\Vy}{\vec{y}}
\newcommand{\Vx}{\vec{x}}
\newcommand{\Ve}{\vec{e}}
\newcommand{\Vw}{\vec{w}}
\newcommand{\Vt}{\vec{\theta}}
\newcommand{\SR}{\mathcal{R}}

\let\emph\relax % there's no \RedeclareTextFontCommand
\DeclareTextFontCommand{\emph}{\color{red}\em}
\setbeamertemplate{headline}{}
\setbeamertemplate{navigation symbols}{}

%%%%%%%%%%%%%%% Section标题页 %%%%%%%%%%%%%%%%%%%%%%%
\AtBeginSection[]{
  \begin{frame}
  \vfill
  \centering
  \begin{beamercolorbox}[sep=8pt,center,shadow=true]{title}
    \usebeamerfont{title}\insertsectionhead\par%
  \end{beamercolorbox}
  \vfill
  \end{frame}
}
%%%%%%%%%%%%%%% 正文开始 %%%%%%%%%%%%%%%%%%%%%%%

\title{第7讲:支持向量机}
\subtitle{带约束的最优化问题}
\author{熊耀华}
\institute{交通工程系}

\begin{document}

\begin{frame}
    \titlepage
\end{frame}

\begin{frame}
  \frametitle{Logistic回归与支持向量机}

  支持向量机(Support Vector Machine)和Logistic回归一样,是常见的分类模型。两者的思路也类似。

  Logistic回归模型分为两步
  \begin{description}
    \item[打分] 用一个线性函数将自变量$\Vx$映射成分数$s=\Vw^T\Vx+b$
    \item[分类] 用Logsitic函数$\sigma$将分数$s$映射成隶属概率$p=\sigma(s)$
  \end{description}

  支持向量机采用了不同的分类方式
\end{frame}

\begin{frame}
  \frametitle{支持向量机的分类方式}
  支持向量机规定,对于数据$\Vx_i$,分数
  \[s_i=\Vw^T\Vx+b\]
  分类直接由分数决定
  \[t_i=\left\{\begin{array}{rl}
    1 & \text{当}s_i\ge 1\\
    -1 & \text{当}s_i\le -1\\
  \end{array}\right.
  \]
  因变量$t_i$的两个取值$\{-1,1\}$分别代表两个类别。
\end{frame}

\begin{frame}
  \frametitle{参数$\Vw$的约束条件}
  对于$m$个数据点构成的训练集
  $$\{(\Vx_i,t_i):i=1,\ldots,m\}$$
  为保证模型与数据吻合,参数$\Vw$必须满足以下\emph{约束}
  \begin{equation}\left.
    \begin{array}{ll}
    \Vw^T\Vx_i+b\ge1 & \text{当}t_i=1\\
    \Vw^T\Vx_i+b\le-1 & \text{当}t_i=-1
    \end{array}\right\}\quad\text{对所有$i$成立}
  \end{equation}
\end{frame}

\begin{frame}
  \frametitle{约束条件的缩写}
  参数$\Vw$约束条件的两种情况可以缩写成一个公式
  \begin{equation}
    t_i(\Vw^T\Vx_i+b)\ge1\quad\text{对所有$i$成立}
  \end{equation}

  注意:这个缩写成立的关键在于两个类别用$\{-1,1\}$表示;
  如果用$\{0,1\}$则不成立。
\end{frame}

\begin{frame}
  \frametitle{最优参数}
  满足上述约束条件的参数$\Vw$很可能有多个,需要选出\emph{最优}的一个。SVM的最优标准是$\|\Vw\|$最小,得到带约束的最优化问题
  \begin{equation}
    \begin{aligned}
    &\min_{\Vw,b} \|\Vw\|\\
    \text{满足}&\\
    &t_i(\Vw^T\Vx_i+b)\ge1\quad\text{$i=1,\ldots,m$}
  \end{aligned}
\end{equation}
\end{frame}

\begin{frame}
  \frametitle{$\Vw$最小的意义}
  给定具体参数$\Vw,b$后,SVM模型把数据空间$\Vx$分成三个部分
  \begin{description}
    \item[第一类] $\Vw^T\Vx+b\ge1$ 
    \item[第二类] $\Vw^T\Vx+b\le-1$ 
    \item[\emph{边界区}] $-1<\Vw^T\Vx+b<1$ 
  \end{description}

  边界区的宽度由$\|\Vw\|$控制,$\|\Vw\|$越小边界区越宽。
\end{frame}
\end{document}