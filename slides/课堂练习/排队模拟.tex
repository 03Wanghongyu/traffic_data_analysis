\documentclass[12pt]{article}
\usepackage{amsmath}
\usepackage{ctex}
\setsansfont{Noto Sans CJK SC Light}
\setCJKsansfont[ItalicFont=Kaiti SC]{Noto Sans CJK SC Light}
\setCJKmainfont{Noto Serif CJK SC Light}
\usepackage{graphicx}
\graphicspath{ {./images/} }
\usepackage{geometry}
 \geometry{
 a4paper,
 total={170mm,257mm},
 left=20mm,
 top=20mm,
 right=20mm,
 }
\DeclareEmphSequence{\bfseries,\itshape,\upshape}
\makeatletter
\renewcommand\maketitle{
\begin{center}
{\LARGE \bfseries \@title}\\[2ex]
\end{center}
}
\makeatother

\title{《交通仿真》课堂练习}
\begin{document}
\maketitle

\section{检查站仿真}

某检查站坐落在某工业园区入口,仅有\emph{一条}检查通道。该入口每天车辆到达率随时间变化,
早上8:00--9:00是高峰期,车辆到达较快,之后减慢;\emph{平均到达率}如下表所示:
\begin{table}[htbp]
    \begin{center}
    \begin{tabular}{ c c c }
        时间段 & 平均到达率 \\
        \hline
        8:00--9:00 & 10辆/小时 \\
        9:00--12:00 & 3辆/小时 \\
    \end{tabular}
    \end{center}
    \caption{车辆平均到达率}
\end{table}

\noindent 同时假设车辆到达符合\emph{柏松过程}。车辆\emph{检测时间}固定为12分钟,
完成检查之后才能离开。

对该场景进行离散事件仿真,并\emph{绘制}累计到达、离开曲线图。根据该图\emph{回答}下列问题
\begin{enumerate}
    \item 仿真时段8:00--12:00内同时最多有几辆车排队?
    \item 到12:00时队列是否会消散?
    \item 排队车辆的最大排队等待时间是多少?
    \item 平均每辆车的排队等待时间时多少?
\end{enumerate}

\emph{提示}:
\begin{itemize}
    \item 符合柏松过程的车流,车头时距符合负指数分布
    \item 负指数分布样本可用公式$y=f^{-1}(x)$生成。其中$f$是负指数分布的\emph{累积分布函数},
    $x$是$[0,1]$上均匀分布样本,$y$是负指数分布样本
    \item 本练习中使用的$[0,1]$均匀分布样本如下
    \begin{equation*}
        \begin{split}
    [0.59, 0.94, 0.97, 0.59, 0.24, 0.5 , 0.89, 0.16, 0.78, 0.98, \\
    0.42, 0.03, 0.35, 0.28, 0.29, 0.55, 0.81, 0.78, 0.87, 0.1 ]
        \end{split}
    \end{equation*}
\end{itemize}
\end{document}