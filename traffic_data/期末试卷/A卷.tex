\documentclass[11pt]{article}

\usepackage{ctex}
\setsansfont{Noto Sans CJK SC Light}
\setCJKsansfont[ItalicFont=Kaiti SC]{Noto Sans CJK SC Light}
\setCJKmainfont{Noto Serif CJK SC Light}
\usepackage{graphicx}
\graphicspath{ {./images/} }
\usepackage{multicol}
\usepackage{enumitem}
\usepackage{lastpage}
\usepackage{fancyhdr}
\renewcommand{\headrulewidth}{0pt}
\fancyhead{}
\cfoot{第{\thepage}页,共\pageref{LastPage}页}
\pagestyle{fancy}
\usepackage{geometry}
 \geometry{
 a4paper,
 total={170mm,257mm},
 left=40mm,
 top=20mm,
 right=20mm,
 }
\usepackage{titlesec}
\renewcommand\thesection{\chinese{section}}
\titleformat{\section}
  {\normalfont\large\bfseries}{\thesection.}{1em}{}
  
\DeclareEmphSequence{\bfseries,\itshape,\upshape}

\newif\ifsolution

%%% 是否显示答案
%\solutiontrue
\solutionfalse

\newcounter{question}
\newenvironment{choice}[2][X]
    {\refstepcounter{question}
        \textbf{\thequestion.} #2
        \hfill\mbox{[\ifsolution{#1}\else\quad\fi]}
        \begin{multicols}{2}
        \begin{enumerate}[label=(\Alph*)]
    }
    {
        \end{enumerate}
        \end{multicols}
    }
\newenvironment{question}[1][XXX]
    {\refstepcounter{question}
        \textbf{\thequestion.}\def\thesolution{#1}
    }
    {}
\usepackage{comment}
\ifsolution
    \newenvironment{solution}[1][10cm]{答案:}{}
\else
    \usepackage{environ}
    \NewEnviron{hide}{}
    \newenvironment{solution}[1][10cm]{答案:\vspace{#1}\hide}
    {\endhide}
\fi


\makeatletter
\renewcommand\maketitle{
\begin{center}
{\LARGE \bfseries \@title}\\[2ex]
\end{center}
}
\makeatother

\title{西南交通大学2021--2022学年第(II)学期考试试卷A}

\usepackage{eso-pic}
\usepackage{dashrule}
\newcommand\BackgroundPicture{
   \put(0,0){\rotatebox{90}{
     \parbox[b][3.5cm]{\paperheight}{
        \vspace{1cm}
        \hfil \textbf{班级}:\underline{\hspace{5cm}}%
        \hfil \textbf{学号}:\underline{\hspace{5cm}}%
        \hfil \textbf{姓名}:\underline{\hspace{5cm}}\hfil
        \vfill
        {\tiny \hfil 密封装订线\hfil 密封装订线\hfil 密封装订线}
        
        \hdashrule[0.5ex][c]{\linewidth}{0.6pt}{1.5mm}
    }}}}
\AddToShipoutPicture*{\BackgroundPicture}

%%%%%%%%%%%%%%%%%%%%%%%%%%%%%%%%%%
\begin{document}
\maketitle
课程代码:\underline{TRAL008012}%
\hfill 课程名称:\underline{交通数据分析}%
\hfill 考试时间:\underline{120分钟}

\begin{center}
\begin{tabular}{|p{0.6cm}|p{0.6cm}|p{0.6cm}|p{0.6cm}|p{0.6cm}|p{0.6cm}|p{0.6cm}|p{0.6cm}|p{0.6cm}|p{0.6cm}|}
    \hline
    1 & 2 & 3 & 4 & 5 & 6 & 7 & 8 & 9 & 10\\
    \hline
    & & & & & & & & &  \\
    \hline
    11 & 12 & 13 & 14 & 15 & 16 & 17 & 18 & 19 & 20\\
    \hline
    & & & & & & & & &  \\
    \hline
    21 & 22 & 23 & 24 & 25 & 26 & 27 & 28 & 29 & 30\\
    \hline
    & & & & & & & & &  \\
    \hline
\end{tabular}
\end{center}
\vspace{2ex}
\hfil 阅卷教师签字:\underline{\hspace{8cm}} \hfil

\section{选择题\textmd{(20小题,每小题3分,共60分)}}

\begin{choice}[B]{交通领域可使用多种形式的原始数据,以下常见数据形式中包含\emph{信息最丰富}的是
    哪一个}
    \item 感应线圈检测器
    \item 车辆GPS轨迹数据
    \item 交叉口车牌识别数据
    \item OD调查数据
\end{choice}

\begin{choice}[C]{综合考虑成本和效率,\emph{最适合}搜集中等以上规模城市整体交通状况的数据采集手段是}
    \item 出租车GPS定位
    \item 交叉口车牌识别
    \item 手机基站定位
    \item 问卷调查
\end{choice}

\begin{choice}[A]{从数据中寻找两个\emph{连续变量}之间函数关系的模型称为}
    \item 回归模型
    \item 分类模型
    \item 聚类模型
    \item 滤波模型
\end{choice}

\begin{choice}[A]{对于\emph{线性回归}模型$y=w\cdot x+b$,损失函数$L$的定义是}
    \item $L=\sum |y-w\cdot x-b|$
    \item $L=\sum (y-w\cdot x-b)$
    \item $L=\sum (y-w\cdot x-b)^2$
    \item $L=\sum (y-w\cdot x-b)^3$
\end{choice}

\begin{choice}[D]{机器学习模型参数\emph{过拟合}现象是指模型损失函数}
    \item 在训练集上大,在测试集上小
    \item 在训练集上大,在测试集上大
    \item 在训练集上小,在测试集上小
    \item 在训练集上小,在测试集上大
\end{choice}

\begin{choice}[C]{线性回归\emph{正则化}的目的是}
    \item 减少参数数量
    \item 增加参数数量
    \item 缩小参数取值范围
    \item 扩大参数取值范围
\end{choice}

\begin{choice}[A]{某数据集包含10000个数据点,从中抽取5000个数据点作为训练集的正确方式是}
    \item 随机抽取5000个数据点
    \item 顺序抽取前5000个数据点
    \item 顺序抽取后5000个数据点
    \item 抽取自变量取值最小的5000个数据点
\end{choice}

\begin{choice}[A]{Logistic回归中用到的映射函数是}
    \item $\sigma(x)=\frac{e^x}{1+e^x}$
    \item $\sigma(x)=\frac{1}{1+e^x}$
    \item $\sigma(x)=\frac{e^x+1}{e^x}$
    \item $\sigma(x)=\frac{e^x+1}{1}$
\end{choice}

\begin{choice}[D]{Logistic回归的损失函数称为}
    \item 均方差函数
    \item 0-1损失函数
    \item 铰链损失函数
    \item 交叉熵损失函数
\end{choice}

\begin{choice}[B]{支持向量机模型中\emph{支持向量}是指}
    \item 满足$y_i(w\cdot x_i+b)> 1$的数据点
    \item 满足$y_i(w\cdot x_i+b)= 1$的数据点
    \item 满足$y_i(w\cdot x_i+b)= 0$的数据点
    \item 满足$y_i(w\cdot x_i+b)> 0$的数据点
\end{choice}

\begin{choice}[D]{以下聚类方法中\emph{不需要}直接指定类别数量的是}
    \item K-平均聚类
    \item K-Medoids聚类
    \item 高斯混合聚类
    \item DBSCAN聚类
\end{choice}

\begin{choice}[A]{高斯混合模型公式如下
    \[f_M(x)=\sum_{i=1}^k\alpha_i\cdot
    \frac{1}{2\pi\sqrt{|\Sigma|}}e^{-\frac{1}{2}
    (y-\mu)^T\Sigma^{-1}(y-\mu)}\]
    其中混合系数是
    }
    \item $\alpha_i$
    \item $\Sigma$
    \item $y$
    \item $\mu$
\end{choice}

\begin{choice}[B]{神经网络的\emph{深度}是指}
    \item 神经元总数
    \item 层数
    \item 输入属性数量
    \item 输出属性数量
\end{choice}

\begin{choice}[D]{单层神经网络的局限性表现在\emph{不能拟合}}
    \item 与门AND
    \item 或门OR
    \item 非门NOT
    \item 异或门XOR
\end{choice}

\begin{choice}[B]{当前图像处理中最常用的神经网络形式是}
    \item 分层神经网络
    \item 卷积神经网络
    \item 循环神经网络
    \item LSTM
\end{choice}

\begin{choice}[A]{神经网络常用激活函数形式中\emph{不连续}的是}
    \item ReLU
    \item Sigmoid
    \item Tanh
    \item Softmax
\end{choice}

\begin{choice}[C]{应用贝叶斯公式时\emph{主要问题}在于构造}
    \item 先验概率
    \item 后验概率
    \item 似然概率
    \item 数据全概率
\end{choice}

\begin{choice}[B]{Kalman滤波模型中描述观测值$z$与系统状态值$x$之间关系的矩阵$H$称为}
    \item 系统状态转移矩阵
    \item 观测矩阵
    \item 协方差矩阵
    \item 正交矩阵
\end{choice}

\begin{choice}[A]{以下滤波模型中通用性最强的是}
    \item 贝叶斯滤波
    \item G-H滤波
    \item Kalman滤波
    \item 粒子滤波
\end{choice}

\begin{choice}[D]{用于矩阵运算的Python模块是}
    \item matplotlib
    \item pandas
    \item scikit
    \item numpy
\end{choice}

\newpage
\section{简答题\textmd{(2小题,每题5分,共10分)}}

\begin{question}
    从几何角度推导最小二乘法公式
    \[A^T\cdot A\hat{\theta}=A^T\cdot y\]
\end{question}
\begin{solution}
    测试一下答案
    \[\int sin(x)dx\]
\end{solution}

\begin{question}
    比较支持向量机的软边界、核方法两种变形,说明两者的实现原理和适用范围方面的异同。
\end{question}

\begin{solution}
    测试一下答案
    \[\int sin(x)dx\]
\end{solution}

\section{计算题\textmd{(2小题,每题15分,共30分)}}
\begin{question}
    
\end{question}
\end{document}